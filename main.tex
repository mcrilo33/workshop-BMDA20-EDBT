\documentclass[sigconf,edbt]{acmart-edbt2020}

\def\BibTeX{{\rm B\kern-.05em{\sc i\kern-.025em b}\kern-.08em
    T\kern-.1667em\lower.7ex\hbox{E}\kern-.125emX}}

\usepackage{booktabs} % For formal tables
\usepackage[squaren, Gray, cdot]{SIunits}



% Copyright
\setcopyright{rightsretained}

% DOI
\acmDOI{}

% ISBN
\acmISBN{XXX-X-XXXXX-XXX-X}

%Conference
\acmConference[EDBT 2020]{22nd International Conference on Extending Database Technology (EDBT)}{March 30-April 2, 2020}{Copenhagen, Denmark} 
\acmYear{2020}

\settopmatter{printacmref=false, printccs=false, printfolios=false}

\pagestyle{empty} % removes running headers


\begin{document}
\title{Motion ResNet : An efficient data imputation method for spatio-temporal series}
\titlenote{Produces the permission block, and copyright information}
\subtitle{Extended Abstract}
\subtitlenote{The full version of the author's guide is available as
  \texttt{acmart.pdf} document}
  

\author{Mathieu Crilout}
\authornote{}
\orcid{}
\affiliation{%
  \institution{LIP6, Sorbonne University}
  \streetaddress{}
  \city{Paris} 
  \state{France} 
  \postcode{}
}
\email{mathieu.crilout@lip6.fr}

\author{Nicolas Baskiotis}
\authornote{}
\affiliation{%
  \institution{LIP6, Sorbonne University}
  \streetaddress{}
  \city{Paris} 
  \state{France} 
  \postcode{}
}
\email{nicolas.baskiotis@lip6.fr}

\author{Vincent Guigue}
\authornote{}
\affiliation{%
  \institution{LIP6, Sorbonne University}
  \streetaddress{}
  \city{Paris} 
  \country{France}}
\email{vincent.guigue@lip6.fr}

% The default list of authors is too long for headers}
% \renewcommand{\shortauthors}{B. Trovato et al.}
\renewcommand{\shortauthors}{}


\begin{abstract}

Real problems have to deal with budget constrained, bandwith limitations and random corruptions during data collection process. It leads to various problems, mostly wrong or incomplete data or datasets with sub-sampled measures.

Various techniques are used to infer missing data : like ignoring incomplete sample, or replacing missing values with some interpolation techniques or statistical models.

More recently, more sophisticate methods using neural nets have been introduced showing promising results.
Moreover, there has been a growing interest in using neural nets as an equations approximator, seeking accurate reconstruction of the data instead of the exact recovery of the equation system.

In this context, this paper introduces a novel method to reconstruct GPS trajectories using equations of motions as architecture guidelines.
It leads to a new recurrent neural net architecture with residual connections called Motion ResNet showing promising results, using both spatial and temporal informations across differents trajectories to impute missing data, with the principal advantage that it can easily take account of other useful informations as, for instance, weather data, traffic context, or driver identity.
\footnote{This is an abstract footnote}
\end{abstract}

%
% % The code below should be generated by the tool at
% % http://dl.acm.org/ccs.cfm
% % Please copy and paste the code instead of the example below. 
% %
% \begin{CCSXML}
% <ccs2012>
%  <concept>
%   <concept_id>10010520.10010553.10010562</concept_id>
%   <concept_desc>Computer systems organization~Embedded systems</concept_desc>
%   <concept_significance>500</concept_significance>
%  </concept>
%  <concept>
%   <concept_id>10010520.10010575.10010755</concept_id>
%   <concept_desc>Computer systems organization~Redundancy</concept_desc>
%   <concept_significance>300</concept_significance>
%  </concept>
%  <concept>
%   <concept_id>10010520.10010553.10010554</concept_id>
%   <concept_desc>Computer systems organization~Robotics</concept_desc>
%   <concept_significance>100</concept_significance>
%  </concept>
%  <concept>
%   <concept_id>10003033.10003083.10003095</concept_id>
%   <concept_desc>Networks~Network reliability</concept_desc>
%   <concept_significance>100</concept_significance>
%  </concept>
% </ccs2012>  
% \end{CCSXML}
% 
% \ccsdesc[500]{Computer systems organization~Embedded systems}
% \ccsdesc[300]{Computer systems organization~Redundancy}
% \ccsdesc{Computer systems organization~Robotics}
% \ccsdesc[100]{Networks~Network reliability}


% \keywords{ACM proceedings, \LaTeX, text tagging}

%% A "teaser" image appears between the author and affiliation
%% information and the body of the document, and typically spans the
%% page.
%%\begin{teaserfigure}
%% \includegraphics[width=\textwidth]{sampleteaser}
%% \caption{Seattle Mariners at Spring Training, 2010.}
%% \label{fig:teaser}
%%\end{teaserfigure}

\maketitle

\input{samplebody-conf}

%%
%% The next two lines define the bibliography style to be used, and
%% the bibliography file.
\bibliographystyle{ACM-Reference-Format}
\bibliography{sample-base}

%%
%% If your work has an appendix, this is the place to put it.
%% Please note that all the content must fit within the page limits, including any appendices.
%\appendix
%
%\section{Research Methods}
% ...

\bibliographystyle{ACM-Reference-Format}
\bibliography{bibfile}
\end{document}
\endinput
